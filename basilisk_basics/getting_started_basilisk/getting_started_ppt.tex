\documentclass[12pt]{beamer}

% including packages
\usepackage{xcolor}

% including and setting up hyperref package
\usepackage{hyperref}
\hypersetup{colorlinks=true, linkcolor=gray, urlcolor=cyan}

\title{How to get started in basilisk CFD solver}
\author{Jishnu Goswami}
\institute{\small{\textcolor{gray}{Physics of Fluids and Soft Matter, Manchester Centre for Nonlinear Dynamics, Department of Physics and Astronomy, University of Manchester, Oxford Rd, UK}}}

\date{\underline{Last Updated:} \today}

\begin{document}
\maketitle

% Frame-1
\begin{frame}{A few points}
\begin{itemize}
\item Basilisk works on adaptive Cartesian grids. \textcolor{gray}{(something I don't know anything about right now)}
\vspace{5mm}
\item Basilisk is written in a variant of c language.
\vspace{5mm}
\item The prerequisites for learning basilisk are the following -
\begin{enumerate}
\vspace{3mm}
\item A good knowledge of \underline{C Programming Language}
\vspace{3mm}
\item A good hand in shell scripting
\end{enumerate}
\end{itemize}
\end{frame}

% Frame-2
\begin{frame}
\begin{itemize}
\item The meaning of installing basilisk into your computing system is to download the source codes from a remotely hosted repository.
\vspace{3mm}
\item There are two ways of getting the source codes of basilisk
\vspace{3mm}
\begin{enumerate}
\item Downloading the source code using \textit{darcs}. The principal benefit here is that this is version controlled.
\item Downloading the source code via \textit{tarball}
\end{enumerate}
\vspace{3mm}
\item The basilisk code is hosted at \href{https://darcs.net/Using}{darcs}. This is an open source version control system. So a little familiarity with this can be helpful.
\vspace{3mm}
\item To run the basilisk codes, we need a c99 compliant compiler. \textcolor{gray}{I am not really comfortable about compilers and different related versions of compilers present. I should look into it in a great detail. Important for GTA stuff too.}
\end{itemize}
\end{frame}

% Frame-3
\begin{frame}{Numerical Methodology}
\begin{itemize}
\item The time-integration step for solving saint-venant equation involves a generic predictor-corrector nummerical scheme. \textcolor{gray}{I need to look into this predictor-corrector method to better understand the numerical schemes at play}
\vspace{5mm}
\item In this example, the issues with \textit{compilers} and \textit{linkers} are introduced. \textcolor{gray}{The basic concepts of compilers and linkers are not clear to me.}
\end{itemize}
\end{frame}
\end{document}